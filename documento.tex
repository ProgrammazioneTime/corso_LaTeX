\documentclass[a4paper,12pt,oneside]{book}

\usepackage[T1]{fontenc}
\usepackage[utf8]{inputenc}
\usepackage[italian]{babel}
\usepackage{geometry}
\geometry{a4paper, top = 3cm, bottom = 3cm, left = 3.5cm, right = 3.5cm, heightrounded, bindingoffset = 5mm}
\pagestyle{plain}
\usepackage{xcolor}
%\usepackage{erewhon}
\usepackage{quoting}
\usepackage[colorlinks]{hyperref}
\hypersetup{hidelinks}
\usepackage{amsmath}
\usepackage{esint}
\usepackage{amsfonts}
\usepackage{amsthm}
\usepackage{graphicx}
%\graphicspath{./nomecartella}
\usepackage{sidecap}
\usepackage{subfig}
\usepackage{wrapfig}
\usepackage{array}
\usepackage{multirow}
\usepackage[autostyle,italian=guillemets]{csquotes}
%\usepackage[backend=biber]{biblatex}
%\usepackage[bibstyle=authortitle,citestyle=verbose-trad1]{biblatex}
%\usepackage[style=alphabetic]{biblatex}
\usepackage[style=authoryear-comp]{biblatex}
%\usepackage[bibstyle=authoryear,citestyle=authoryear-comp]{biblatex}
\usepackage{imakeidx}


\title{Corso YouTube per il\\linguaggio \LaTeX}
\author{Programmazione Time \thanks{Un ringraziamento speciale ai miei iscritti.}}
\date{22 settembre 2021}	% date{}


%\setcounter{secnumdepth}{3}
%\setcounter{tocdepth}{1}
\theoremstyle{plain}
\newtheorem{teorema}{Teorema}[section]
\addbibresource{database.bib}
\makeindex
\newcommand{\predefinito}{\colorbox{orange}{Questo comando è predefinito}}
\newcommand{\comando}[1]{\textit{\textbf{#1}}}
\newenvironment{corsivo}{\begin{itemize}\itshape}{\end{itemize}}


\begin{document}
	\begin{titlepage}
		\maketitle
	\end{titlepage}

	\frontmatter
	\tableofcontents
	\listoffigures
	\listoftables
	\mainmatter
	
	\part{Prima parte}
	\chapter{Primo capitolo}
	Questo è il primo capitolo.
	\section{Prima sezione}
	Questa è la prima sezione.
	\subsection{Prima sottosezione}
	Questa è la prima sottosezione.
	\subsubsection{Prima sotto-sottosezione (non indicizzata)}
	Dalla sotto-sottosezione in poi non vengono più indicizzata nell'indice.
	\paragraph{Primo paragrafo}
	Questo è un paragrafo.
	\subparagraph{Primo sottoparagrafo}
	Primo documento in \LaTeX creato grazie al video di Programmazione Time.
	
	
	\section*{Non numerata}
	\addcontentsline{toc}{section}{Non numerata}
	Questa sezione non è numerata e non è presente nell'indice poiché si è aggiunto l'asterisco.
	
	
	
	\chapter{Gestione del testo}
	\section{Capoversi}
	\subsection{Metodi di creazione}
	\subsubsection{Metodo 1:}
	Il \emph{capoverso}\index{capoverso} segnala l'inizio di un nuovo argomento.
	
	Questo è un capoverso generato saltando una riga nell'editor.
	
	\subsubsection{Metodo 2:}
	Il capoverso segnala l'inizio di un nuovo argomento.
	\par Questo è un capoverso generato con l'utilizzo del comando \textbf{$\backslash$par}.
	
	\subsubsection{Metodo 3:}
	Il capoverso segnala l'inizio di un nuovo argomento.\newline
	\newline
	Questo è un capoverso senza rientro generato utilizzando il doppio $\backslash\backslash$.\\
	\\
	\par Questo è un capoverso con rientro generato utilizzando \textbf{$\backslash$par} e il doppio $\backslash\backslash$.
	
	
	\subsection{Altro sui capoversi}
	\subsubsection{Eliminazione del rientro}
	Il capoverso segnala l'inizio di un nuovo argomento.
	
	\noindent Questo è un capoverso senza il rientro con il comando \textbf{$\backslash$noindent}.
	
	\subsubsection{Gestione dello spazio verticale}
	Il capoverso segnala l'inizio di un nuovo argomento.
	
	\vspace{15pt} Questo è un capoverso con spazio gestito da \textbf{$\backslash$vspace}.
	
	
	\section{Decorazione testo}\label{sec:testo}
	\emph{Stile EVIDENZIATO tramite il comando}: $\backslash$emph.\\
	{\em Stile EVIDENZIATO tramite la dichiarazione}: $\backslash$em.\\
	\\
	\textit{Stile CORSIVO tramite il comando}: $\backslash$textit.\\
	{\itshape Stile CORSIVO tramite la dichiarazione}: $\backslash$itshape.\\
	\\
	\textsc{Stile MAIUSCOLETTO tramite il comando}: $\backslash$textsc.\\
	{\scshape Stile MAIUSCOLETTO tramite la dichiarazione}: $\backslash$scshape.\\
	\\
	\textbf{Stile GRASSETTO tramite il comando}: $\backslash$textbf.\\
	{\bfseries Stile GRASSETTO tramite la dichiarazione}: $\backslash$bfseries.\\
	\\
	\textsl{Stile INCLINATO tramite il comando}: $\backslash$textsl.\\
	{\slshape Stile INCLINATO tramite la dichiarazione}: $\backslash$slshape.\\
	\\
	\textrm{Stile TONDO tramite il comando}: $\backslash$textrm.\\
	{\rmfamily Stile TONDO tramite la dichiarazione}: $\backslash$rmfamily.\\
	\\
	\textsf{Stile SENZA GRAZIE tramite il comando}: $\backslash$textsf.\\
	{\sffamily Stile SENZA GRAZIE tramite la dichiarazione}: $\backslash$sffamily.\\
	\\
	\texttt{Stile MACCHINA PER SCRIVERE tramite il comando}: $\backslash$texttt.\\
	{\ttfamily Stile MACCHINA PER SCRIVERE tramite la dichiarazione}: $\backslash$ttfamily.\\
	\\
	\underline{Testo SOTTOLINEATO tramite il comando}: $\backslash$underline.\\
	\textcolor{blue}{Questa frase è COLORATA tramite il comando}: $\backslash$textcolor.\\
	\colorbox{yellow}{Questa frase è EVIDENZIATA tramite il comando}: $\backslash$colorbox.\\
	\\
	\textit{\textbf{Questa frase è sia in CORSIVO sia in GRASSETTO}}.\\
	\underline{{\ttfamily Questa frase è sia in MACCHINA DA SCRIVERE sia SOTTOLINEATA}}.
	
	
	\section{Dimensione del font}
	{\tiny Dimensione del testo TINY}.\\
	{\scriptsize Dimensione del testo SCRIPTSIZE}.\\
	{\footnotesize Dimensione del testo FOOTNOTESIZE}.\\
	{\small Dimensione del testo SMALL}.\\
	{\normalsize Dimensione del testo NORMALSIZE}.\\
	{\large Dimensione del testo large}.\\
	{\Large Dimensione del testo Large}.\\
	{\LARGE Dimensione del testo LARGE}.\\
	{\huge Dimensione del testo huge}.\\
	{\Huge Dimensione del testo Huge}.
	
	
	
	\section{Note}
	\subsection{Al margine}
	Tramite il comando \textbf{$\backslash$marginpar} è possibile creare una nota\marginpar{Questa è una nota al margine.} al margine del foglio.
	
	\subsection{Piè di pagina}
	Tramite il comando \textbf{$\backslash$footnote} si creano note\footnote{Questa è una nota al piè di pagina.} in fondo al foglio. Ogni volta le note vengono numerate così da poterle distinguere\footnote{Questa è una seconda nota.}.
	
	
	\section{Ambienti testuali}
	\subsection{Elenchi}
	Elenco puntato:
	\begin{itemize}
		\item Primo punto.
		\item Secondo punto.
		\begin{itemize}
			\item Sotto elenco puntato.
			\item Secondo sotto punto.
			\begin{itemize}
				\item Terzo livello di elenco puntato.
				\begin{itemize}
					\item Quarto livello di elenco puntato.
					\item [+] Punto personalizzato.
					\item [>] Punto personalizzato.
				\end{itemize}
			\end{itemize}
		\end{itemize}
		\item [@] Terzo punto.
	\end{itemize}
	Elenco numerato:
	\begin{enumerate}
		\item Primo punto.
		\item Secondo punto.
		\begin{enumerate}
			\item Sotto elenco numerato.
			\item Secondo sotto punto.
			\begin{enumerate}
				\item Terzo livello di elenco numerato.
				\begin{enumerate}
					\item Quarto livello di elenco numerato.
					\item Nuovo elemento.
				\end{enumerate}
				\item Altro elemento.
			\end{enumerate}
		\end{enumerate}
	\end{enumerate}
	\newpage
	Elenco misto:
	\begin{itemize}
		\item Primo punto.
		\item Secondo punto.
		\begin{enumerate}
			\item Sotto elenco numerato.
			\item Secondo sotto punto.
			\begin{itemize}
				\item Terzo livello di elenco puntato.
				\begin{enumerate}
					\item Quarto livello di elenco numerato.
					\item Nuovo elemento.
				\end{enumerate}
				\item Altro elemento.
			\end{itemize}
		\end{enumerate}
	\end{itemize}


	\subsection{Allineamento del capoverso}
	\begin{flushleft}
		Questo testo è allineato a sinistra tramite \textbf{flushleft}.\\
		Definizione del termine \textit{informatica} presa da Wikipedia: l'informatica è la scienza che si occupa del trattamento dell'informazione mediante procedure automatizzate, avendo in particolare per oggetto lo studio dei fondamenti teorici dell'informazione, della sua computazione a livello logico e delle tecniche pratiche per la sua implementazione e applicazione in sistemi elettronici automatizzati detti quindi \emph{sistemi informatici}\index{sistemi informatici}; come tale è una disciplina fortemente connessa con la logica matematica, l'automatica, l'elettronica e anche l'elettromeccanica.
	\end{flushleft}
	
	\begin{flushright}
		Questo testo è allineato a destra tramite \textbf{flushright}.\\
		Definizione del termine \textit{informatica} presa da Wikipedia: l'informatica è la scienza che si occupa del trattamento dell'informazione mediante procedure automatizzate, avendo in particolare per oggetto lo studio dei fondamenti teorici dell'informazione, della sua computazione a livello logico e delle tecniche pratiche per la sua implementazione e applicazione in sistemi elettronici automatizzati detti quindi sistemi informatici; come tale è una disciplina fortemente connessa con la logica matematica, l'automatica, l'elettronica e anche l'\emph{elettromeccanica}\index{elettromeccanica}.
	\end{flushright}
	
	\begin{center}
		Questo testo è allineato al centro tramite \textbf{center}.\\
		Definizione del termine \textit{informatica} presa da Wikipedia: l'informatica è la scienza che si occupa del trattamento dell'informazione mediante procedure automatizzate, avendo in particolare per oggetto lo studio dei fondamenti teorici dell'informazione, della sua computazione a livello logico e delle tecniche pratiche per la sua implementazione e applicazione in sistemi elettronici automatizzati detti quindi sistemi informatici; come tale è una disciplina fortemente connessa con la logica matematica, l'automatica, l'elettronica e anche l'elettromeccanica.
	\end{center}
	
	
	\subsection{Citazioni in display}
	Giacomo Leopardi è stato un poeta, filosofo, scrittore, filologo italiano.
	\begin{quoting}
		``La vita e l'assoluta mancanza d'illusione, e quindi di speranza, sono cose contraddittorie.''
	\end{quoting}
	È ritenuto il maggior poeta dell'Ottocento italiano e una delle più importanti figure della letteratura mondiale.
	
	
	\subsection{Poesie}
	Poesia \textit{A Zacinto} di Ugo Foscolo:
	\begin{verse}
		Né più mai toccherò le sacre sponde\\
		Ove il mio corpo fanciulletto giacque,\\
		Zacinto mia, che te specchi nell'onde\\
		Del greco mar, da cui vergine nacque
		
		Venere, e fea quelle isole feconde\\
		Col suo primo sorriso, onde non tacque\\
		Le tue limpide nubi e le tue fronde\\
		L’inclito verso di colui che l'acque
		
		Cantò fatali, ed il diverso esiglio\\
		Per cui bello di fama e di sventura\\
		Baciò la sua petrosa Itaca Ulisse.
		
		Tu non altro che il canto avrai del figlio,\\
		O materna mia terra; a noi prescrisse\\
		Il fato illacrimata sepoltura.
	\end{verse}
	
	
	\subsection{Riferimenti incrociati}
	Questo è un riferimento alla sezione \ref{sec:testo} (pagina \pageref{sec:testo}).
	
	
	\subsection{Collegamento web}
	Cliccando sul nome \href{https://www.youtube.com/channel/UCDq9FlqxaAZmgoLBgf5KtYA}{Programmazione Time} si verrà reindirizzati al canale YouTube.\\
	\\
	Clicca il link \url{www.google.com} per raggiungere il motore di ricerca Google.
	
	
	
	\chapter{Caratteri speciali e regole d'uso}
	Tra    queste     parole    viene   inserito  un    solo     spazio. Mentre in questo caso ci sono ben cinque\space\space\space\space\space spazi vuoti.\\
	\\
	Tipologie di virgolette:
	\begin{itemize}
		\item  gli `apici';
		\item le ``virgolette inglesi'';
		\item le <<virgolette caporali>>.
	\end{itemize}
	Non si fa uso dei singoli 'apici' o delle classiche "virgolette".\\
	\\
	Tipologie di tratto:
	\begin{itemize}
		\item Trattino - normale;
		\item Tratto -- con doppio trattino;
		\item Lineetta --- con triplo trattino;
		\item Simbolo meno $-$.
	\end{itemize}
	I tre puntini di sospensione \dots{} sono generati dal comando \textbf{$\backslash$dots\{\}}.\\
	\\
	Creazione di un apice mediante \textbf{$\backslash$ap\{\}}: sig.\ap{re} e sig.\ap{ra}.\\
	Creazione di un pedice mediante \textbf{$\backslash$ped\{\}}: A\ped{1}, parola\ped{pedice}.\\
	\\
	Caratteri ``problematici'':
	\begin{itemize}
		\item Il dollaro \$
		\item Il cancelletto \#
		\item La percentuale \%
		\item La `e' commerciale \&
		\item L'accento circonflesso $\hat{}$
		\item Il tilde $\tilde{}$
		\item Il trattino basso \_
		\item Le parentesi graffe \{ \}
		\item La barra rovescia (\textit{backslash}) $\backslash$
	\end{itemize}
	
	
	
	\chapter{Formule matematiche}
	\section{Tipologia di formule}
	\subsection{Formula inline}
	Esistono tre modi per generare la \textbf{formula inline}:
	\begin{enumerate}
		\item Doppio uso del dollaro: $10\cdot 5 = 50$.
		\item Parentesi tonde precedute dal backslash: \(45 / 3 = 15\).
		\item Ambiente matematico \textbf{math}: \begin{math}
			37 - 14 = 23
		\end{math}
	\end{enumerate}
	\LaTeX cerca di comprimere meglio che può $\sum_{x=1}^{10}\frac{x}{5} = 11$ le formule matematiche in linea.
	
	
	\subsection{Formula in display}
	Esistono tre modi per generare la \textbf{formula in display}:
	\begin{enumerate}
		\item Ambiente matematico \textbf{equation}:
		\begin{equation}
			v = \frac{s}{t}
		\end{equation}
		
		\item Parentesi quadre precedute dal backslash $\backslash$[ $\backslash$]:
		\[
		rad = \frac{\pi}{180}\theta
		\]
		
		\item Ambiente matematico \textbf{displaymath}:
		\begin{displaymath}
			\alpha = \frac{\Delta\omega}{\Delta t}
		\end{displaymath}
	\end{enumerate}

	
	\section{Opzioni interessanti}
	\subsection{Affiancare più espressioni}
	Per affiancare più espressioni all'interno di un singolo ambiente si fa uso dei comandi: \textbf{$\backslash$quad} e \textbf{$\backslash$qquad} per spaziare le due formule.
	\[
	y = mx + q\qquad y = \frac{x - q}{m}
	\]
	
	\subsection{Inserire una piccola porzione di testo}
	Per inserire una piccola porzione di testo all'interno dell'ambiente si utilizza \textbf{$\backslash$text}:
	\[
	y = \sqrt{x} \quad\textrm{per $x\geq 0$}
	\]
	
	
	\section{Forme utili}
	\begin{itemize}
		\item Per creare l'esponente si utilizza l'accento circonflesso $\hat{}$ :
		\[
		y = ax^2 + bx + c\qquad y = e^{x - 5}
		\]
		
		\item Per creare gli indici si utilizza il trattino basso \_ :
		\[
		x_1 + x_2 = k\qquad x_{a,\frac{\alpha}{2}}\cdot y_{z,i} = m
		\]
		
		\item Per creare radici si utilizza il comando \textbf{$\backslash$[root]sqrt}:
		\[
		\sqrt{5x + 3}\qquad \sqrt[3]{\frac{2x^2+1}{5x}-1} \qquad \sqrt[\frac{1}{3}]{e^{x+1}}
		\]
		
		\item Per creare una sommatoria si utilizza il comando \textbf{$\backslash$sum\_\{min\}$\hat{}$\{max\}}:
		\[
		\sum_{i=1}^{n}x_i\qquad \sum_{i=1}x_i\qquad \sum^{n}x\qquad \sum x
		\]
		
		\item Per creare una produttoria si utilizza il comando \textbf{$\backslash$prod\_\{min\}$\hat{}$\{max\}}:
		\[
		\prod_{i=1}^{n}x_i\qquad \prod_{i=1}x_i\qquad \prod^{n}x\qquad \prod x
		\]
		
		\item Per creare un limite si utilizza il comando\\
		\textbf{$\backslash$lim\_\{variabile $\backslash$to valore\}}:
		\[
		\lim_{x\to 0}\frac{\sin(x)}{x} = 1\qquad \lim_{x\to 0^-}\frac{1}{x} = -\infty
		\]
		
		\item Per creare un integrale si utilizza il comando \textbf{$\backslash$int\_\{min\}$\hat{}$\{max\}}:
		\[
		\int_{0}^{\alpha}f(x)\,dx \qquad \int_{b}f(x)\,dx \qquad \int^{a}f(x)\,dx \qquad \int f(x)\,dx
		\]
		Integrali multipli e curvilinei con i pacchetti \textbf{amsmath} ed \textbf{esint}:
		\[
		\iint_D f(x,y)\,dx\,dy \qquad \iint g\,dx\,dy \qquad
		\iiint g \,dx\,dy\,dz \qquad \oint f(z)\,dz=2\pi i
		\]
		
		\item Adattamento delle parentesi con \textbf{$\backslash$left} e \textbf{$\backslash$right}:\\
		Senza ridimensionamento
		\[
		(\frac{\sqrt{5x+1}}{\int_{0}^{\alpha}x^3+\frac{e^x}{2}\,dx}) \qquad [\frac{\sqrt{5x+1}}{\int_{0}^{\alpha}x^3+\frac{e^x}{2}\,dx}] \qquad \{\frac{\sqrt{5x+1}}{\int_{0}^{\alpha}x^3+\frac{e^x}{2}\,dx}\}
		\]
		Con ridimensionamento automatico
		\[
		\left(\frac{\sqrt{5x+1}}{\int_{0}^{\alpha}x^3+\frac{e^x}{2}\,dx}\right) \qquad \left[\frac{\sqrt{5x+1}}{\int_{0}^{\alpha}x^3+\frac{e^x}{2}\,dx}\right] \qquad \left\{\frac{\sqrt{5x+1}}{\int_{0}^{\alpha}x^3+\frac{e^x}{2}\,dx}\right\}
		\]
		
		\item Per scrivere sopra o sotto a delle espressioni potete usare entrambe le possibilità:
		\[
		\underbrace{1 + 2, \dots, n}_{\frac{n(n+1)}{2}} + (n+1) \qquad \overbrace{1 + 2, \dots, n}^{\frac{n(n+1)}{2}} + (n+1)
		\]
		\[
		\underset{y=}{ax^2 + bx + c} \qquad \overset{y=}{ax^2 + bx + c}
		\]
		
		\item Per creare sistemi di equazioni su usa l'ambiente matematico \textbf{cases}:
		\[
		\begin{cases}
			x + y = 2\\
			x - y = 0
		\end{cases} \qquad \left\{\!\begin{array}{c}
			x + y = 2\\
			x - y = 0
		\end{array}\right.
		\]
		Creazione di insiemi:
		\[
		\left\{\!\begin{array}{c | c}
			\frac{1}{n^2} & n\in\mathbb{N}
		\end{array}\!\right\}
		\]
	\end{itemize}


	\section{Vettori e matrici}
	\begin{itemize}
		\item Matrice e vettori senza parentesi:
		\[
		\begin{matrix}
			1 & 1 & \dots & 1\\
			\vdots & \vdots & \ddots & \vdots\\
			1 & 1 & \dots & 1
		\end{matrix} \qquad \begin{matrix}
			1 & 1 & \dots & 1
		\end{matrix} \qquad \begin{matrix}
			1\\
			\vdots\\
			1
		\end{matrix}
		\]
		
		\item Matrice e vettori con parentesi tonde:
		\[
		\begin{pmatrix}
			1 & 1 & \dots & 1\\
			\vdots & \vdots & \ddots & \vdots\\
			1 & 1 & \dots & 1
		\end{pmatrix} \qquad \begin{pmatrix}
			1 & 1 & \dots & 1
		\end{pmatrix} \qquad \begin{pmatrix}
			1\\
			\vdots\\
			1
		\end{pmatrix}
		\]
		
		\item Matrice e vettori con parentesi quadre:
		\[
		\begin{bmatrix}
			1 & 1 & \dots & 1\\
			\vdots & \vdots & \ddots & \vdots\\
			1 & 1 & \dots & 1
		\end{bmatrix} \qquad \begin{bmatrix}
			1 & 1 & \dots & 1
		\end{bmatrix} \qquad \begin{bmatrix}
			1\\
			\vdots\\
			1
		\end{bmatrix}
		\]
		
		\item Matrice e vettori con parentesi graffe:
		\[
		\begin{Bmatrix}
			1 & 1 & \dots & 1\\
			\vdots & \vdots & \ddots & \vdots\\
			1 & 1 & \dots & 1
		\end{Bmatrix} \qquad \begin{Bmatrix}
			1 & 1 & \dots & 1
		\end{Bmatrix} \qquad \begin{Bmatrix}
			1\\
			\vdots\\
			1
		\end{Bmatrix}
		\]
		
		\item Matrice e vettori con linea:
		\[
		\begin{vmatrix}
			1 & 1 & \dots & 1\\
			\vdots & \vdots & \ddots & \vdots\\
			1 & 1 & \dots & 1
		\end{vmatrix} \qquad \begin{vmatrix}
			1 & 1 & \dots & 1
		\end{vmatrix} \qquad \begin{vmatrix}
			1\\
			\vdots\\
			1
		\end{vmatrix}
		\]
		
		\item Matrice e vettori con doppia linea:
		\[
		\begin{Vmatrix}
			1 & 1 & \dots & 1\\
			\vdots & \vdots & \ddots & \vdots\\
			1 & 1 & \dots & 1
		\end{Vmatrix} \qquad \begin{Vmatrix}
			1 & 1 & \dots & 1
		\end{Vmatrix} \qquad \begin{Vmatrix}
			1\\
			\vdots\\
			1
		\end{Vmatrix}
		\]
	\end{itemize}


	\section{Raggruppamento e gestione delle formule}
	\subsection{Spezzare formule}
	Per spezzare una formula lunga si utilizza \textbf{multline}:
	\begin{multline}
		z = a + b + c + d\\
		+ e + f + g + h\\
		+ i + l + m + n\\
		+ o + p + q + r\\
		+ s + u + t + v
	\end{multline}
	Per spezzare una formula incolonnandola si utilizza \textbf{split}:
	\[
	\begin{split}
		z &= a + b + c + d\\
		&= e + f + g + h\\
		&= i + l + m + n\\
		&= o + p + q + r\\
		&= s + u + t + v
	\end{split}
	\]
	
	
	\subsection{Raggruppare formule}
	Per raggruppare formule si utilizza \textbf{gather}:
	\begin{gather}
		y = mx + q\\
		y = ax^2 + bx + c\\
		y = \sin(x)
	\end{gather}
	Per raggruppare formule incolonnandole si utilizza \textbf{align}:
	\begin{align}
		y &= mx + q	& y &= ax^2 + bx + c & y &= \sin(x)\\
		x^2 + y^2 + ax + by + c &= 0 & xy &= k & y &= \tan(x)
	\end{align}
	
	
	\section{Teoremi ed enunciati}
	\begin{teorema}[Weierstrass]
		Una funzione $f:Dom(f)\subseteq \mathbb{R}\rightarrow\mathbb{R}$ definita e continua su un insieme compatto (un insieme chiuso e limitato) ammette in esso un massimo e un minimo assoluti.
	\end{teorema}
	\begin{teorema}[Rolle]
		Sia $f:[a,b]\rightarrow\mathbb{R}$ una funzione continua in $[a,b]$ e derivabile in $(a,b)$. Se la funzione assume lo stesso valore agli estremi dell'intervallo, ossia
		\[
		f(a) = f(b)
		\]
		allora esiste almeno un punto $x_0\in(a,n)$ tale che
		\[
		f'(x_0) = 0
		\]
	\end{teorema}
	\begin{proof}
		Dato che sono soddisfatte le ipotesi del Teorema di Weierstrass, sappiamo che la funzione $y = f(x)$ assume in $[a,b]$ un massimo $M$ e un minimo $m$ assoluti.\\
		Ci sono così due possibilità:
		\begin{itemize}
			\item Se il massimo e il minimo assoluto coincidono, ossia $M = m$, allora $y = f(x)$ è costante. Di conseguenza $f'(x) = 0$ per ogni punto $x$ di $(a,b)$ il teorema vale sicuramente.
			
			\item Se invece $m < M$, poiché nell'ipotesi $f(a) = f(b)$, almeno uno dei due valori $m$ oppure $M$ è assunto dalla funzione in un punto $x_0$ interno all'intervallo. Se $f(x_0) = M$, allora $x_0$ è un punto estremante e per il Teorema di Fermat risulta che $f'(x_0) = 0$.
		\end{itemize}
	\end{proof}


	\chapter{Importare immagini}
	Le immagini vengono importate aggiungendo il pacchetto \textbf{graphicx} e utilizzando il comando \textbf{$\backslash$includegraphics[opzioni]\{nome immagine\}}:\\
	\includegraphics[width=0.3\textwidth]{logo}
	
	
	\section{Didascalie laterali}
	\begin{SCfigure}[50][h!]
		\includegraphics[width=0.3\textwidth]{logo}
		\caption{Logo del canale YouTube con didascalia laterale}
		\label{fig:laterale}
	\end{SCfigure}

	
	\newpage
	\section{Oggetti multipli}
	\begin{figure}[h!]
		\centering
		\subfloat[][\textbf{\emph{Logo 1 del canale}}]{\includegraphics[width=0.3\textwidth]{logo}}\quad
		\subfloat[][\textbf{\emph{Logo 2 del canale}}]{\includegraphics[width=0.3\textwidth]{logo}}\\
		\subfloat[][\textbf{\emph{Logo 3 del canale}}]{\includegraphics[width=0.3\textwidth]{logo}}\quad
		\subfloat[][\textbf{\emph{Logo 4 del canale}}]{\includegraphics[width=0.3\textwidth]{logo}}
		\caption{Dentro all'ambiente \textbf{figure} ci sono 4 sotto figure}
		\label{fig:sotto}
	\end{figure}
	
	
	\section{Immagine immersa nel testo}
	\begin{wrapfloat}{figure}{i}{0pt}
		\includegraphics[width=0.3\textwidth]{vecchio}
		\caption{Logo originale del canale}
	\end{wrapfloat}
	Il canale YouTube \textbf{Programmazione Time} è stato creato il 22 settembre del 2016. Inizialmente doveva essere una piattaforma sul quale potessi caricare con facilità i video sul linguaggio C per aiutare i miei compagni di classe, poiché Whatsapp limitava la condivisione e peggiorava la qualità, mentre Telegram lo possedevano in pochi.
	
	Non mi sarei mai aspettato di continuare quella strada presa così per caso, dopotutto non avevo nemmeno chissà quali strumenti di registrazione: un vecchio portatile con Windows XP, connessione internet a 3 Mb/s, il microfono dello smartphone (Samsung Galaxy Young). Tuttavia avevo e ho tanta buona forza di volontà, solo questo mi ha permesso di non fermarmi mai.
	
	Oggigiorno il canale conta più di 17000 iscritti e ha l'obbiettivo di portare corsi universitari (e non) relativi all'ambito informatico. C'è ancora molta strada da compiere, riformare determinati contenuti, condividere ciò che ho appreso al pubblico sperando possa rivelarsi utile.
	
	
	
	\chapter{Creare tabelle}
	L'ambiente predisposto per le tabelle è \textbf{tabular}, dove all'interno delle graffe si specificano i descrittori delle colonne:
	\begin{center}
		\begin{tabular}{l c r}
			Cella 1 & Cella 2 & Cella 3\\
			Cella 4 & Cella 5 & Cella 6\\
			Cella 7 & Cella 8 & Cella 9
		\end{tabular}
	\end{center}
	\begin{center}
		\begin{tabular}{l | c r}
			\hline
			Cella 1 & Cella 2 & Cella 3\\ [0.5ex]
			\hline
			Cella 4 & Cella 5 & Cella 6\\
			\hline
			Cella 7 & Cella 8 & Cella 9\\ [1ex]
			\hline
		\end{tabular}
	\end{center}
	\begin{center}
		\begin{tabular}{ | m{8em} | m{2cm} | m{1cm} | } 
			\hline
			Cella 1 & Cella 2 & Cella 3\\
			\hline
			Cella 4 & Cella 5 & Cella 6\\
			\hline
			Cella 7 & Cella 8 & Cella 9\\
			\hline
		\end{tabular}
	\end{center}


	\section{Unire righe e colonne}
	Per unire le colonne si utilizza il comando \textbf{$\backslash$multicolum}:
	\begin{center}
		\begin{tabular}{| c | c | c |}
			\hline
			\multicolumn{3}{|c|}{Unione} \\
			\hline
			Cella 4 & Cella 5 & Cella 6\\
			\hline
			Cella 7 & Cella 8 & Cella 9\\
			\hline
		\end{tabular}
	\end{center}
	Per unire le righe si utilizza il comando \textbf{$\backslash$multirow}:
	\begin{center}
		\begin{tabular}{| c | c | c |}
			\hline
			\multirow{3}{4em}{Unione} & Cella 2 & Cella 3\\
			& Cella 5 & Cella 6\\
			& Cella 8 & Cella 9\\
			\hline
		\end{tabular}
	\end{center}
	\newpage
	Tabella che appare nell'indice:
	\begin{table}[h!]
		\centering
		\begin{tabular}{|| c | c ||}
			\hline
			Cella 1 & Cella 2\\
			Cella 3 & Cella 4\\
			Cella 5 & Cella 6\\
			Cella 7 & Cella 8\\
			\hline
		\end{tabular}
		\caption{Tabella con doppia linea}
		\label{tab:tabella}
	\end{table}
	
	
	
	\chapter{Bibliografia}
	\section{Automatica}
	Per citare un'opera nel documento si utilizza \textbf{$\backslash$cite}\\(lo schema è \textbf{authoryear-comp}):\\
	\cite{man:sposi}\\
	\cite[1]{man:sposi}\\
	\cite[8-13]{man:sposi}\\
	\cite[vedi][]{man:sposi}\\
	\cite[vedi][1]{man:sposi}\\
	\\
	Ogni contenuto citato apparirà poi nella sezione bibliografica:\\
	\cite{tube:canale}\\
	\cite[vedi][]{tube:canale}\\
	\cite[vedi][20]{corriere:giornale}\\
	\\
	Altra serie di comandi appartenenti a \textbf{biblatex}:
	\begin{itemize}
		\item Usare \textbf{$\backslash$textcite} quando la citazione è parte integrante del discorso: \textcite{man:sposi};
		
		\item Usare \textbf{$\backslash$parencite} per racchiudere la citazione tra parentesi quadre: \parencite{corriere:giornale};
		
		\item Usare \textbf{$\backslash$footcite} quando la citazione dev'essere una nota\footcite{tube:canale};
		
		\item Usare \textbf{$\backslash$supercite} per mettere la citazione in apice: \supercite{man:sposi};
		
		\item Usare \textbf{$\backslash$fullcite} quando si vuole riportare la citazione per intero: \fullcite{man:sposi};
	\end{itemize}
	Per singole parti di citazione:
	\begin{itemize}
		\item Solo autore \citeauthor{tube:canale}, con \textbf{$\backslash$citeauthor}.
		\item Solo anno \citeyear{corriere:giornale}, con \textbf{$\backslash$citeyear}.
	\end{itemize}
		
	\printbibliography
	
	\newpage
	\section{Manuale}
	Per le citazioni manuali si può utilizzare il comando \textbf{$\backslash$cite} come visto in precedenza.

	\begin{thebibliography}{9}
		\bibitem{Ungaretti}{giuseppe:ungaretti}
		Ungaretti, Giuseppe (1978),
		\emph{Lettere dal fronte a Gherardo Marone.}, A. Mondadori, Milano.
		
		\bibitem{eco:tesi}
		Eco, Umberto (1977),
		\emph{Come si fa una tesi di
			laurea}, Bompiani, Milano.
	\end{thebibliography}


	\chapter{Indice analitico}
	La creazione dell'\emph{indice analitico}\index{indice analitico|see{Capitolo 8}} ha bisogno del pacchetto \textbf{imakeidx}, inoltre nel preambolo si deve inserire il comando \textbf{$\backslash$makeindex} e prima della fine del documento inserire \textbf{$\backslash$printidex}.
	
	
	\chapter{Creare nuovi comandi e ambienti}
	\section{Nuovi comandi}
	La creazione di nuovi comandi è dato dal comando \textbf{$\backslash$newcommand} nel preambolo, mentre per ridefinire il comando si usa \textbf{$\backslash$renewcommand}.\\
	\\
	\predefinito\\
	\comando{Questo comando è stato creato unendo altri comandi}.\\
	\\
	\renewcommand{\comando}[1]{\textsc{#1}}
	\comando{È stato modificato il comando usato in precedenza}.
	
	
	\section{Nuovi ambienti}
	La creazione di nuovi ambienti è dato dal comando \textbf{$\backslash$newenviroment} nel preambolo, mentre per ridefinire il comando si usa \textbf{$\backslash$renewenviroment}.
	\begin{corsivo}
		\item Ogni elemento è già in corsivo;
		
		\item Sotto elenco:
		\begin{corsivo}
			\item Continuano a valere le regole viste per gli elenchi;
			
			\item Sotto elemento 2;
		\end{corsivo}
		
		\item Elemento 2;
		
		\item Elemento 3.
	\end{corsivo}
	
	
	
	\printindex
\end{document}